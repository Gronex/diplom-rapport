\documentclass[a4paper,11pt]{book}

\newcommand{\vp}{\left(}
\newcommand{\hp}{\right)}
%%%%%%%%%%%%%%%%
%%% PACKAGES %%%
%%%%%%%%%%%%%%%%
% !TeX spellcheck = en_GB
%\usepackage{maplestd2e}					    	% Til maple input
\usepackage{listings}
\usepackage{float}                                  % Pakke til at placere figurer hvor de fandme skal være!
\usepackage[english]{babel} 			    	        % Engelsk/Dansk ordbog
\usepackage[utf8]{inputenc}							% Understøttelse af æ, ø og å Editor skal være sat op til UTF-8 encoding
\usepackage[T1]{fontenc}                  			% Bruger en rigtig font til dansk output.
\usepackage{amssymb, amsmath}  	    				% Nice equations
%\usepackage{ulem}                                  % Underline pakke
\usepackage{lscape}		    						% Pakke til at lave enkelte landscabe sider (\begin{landscape})
%\usepackage{siunitx}								% Nice cmd til sienheder. Bl.a.: \num{.3e-4}
\usepackage{lmodern} 								% Fontpakke for pdflatex
\usepackage{wrapfig}								% Pakke til ordombrydning
%\usepackage{subfig}	 								% Nice måde at samle flere figurer.
%\usepackage[square,sort]{natbib}
%		\bibpunct{[}{]}{;}{n}{,}{,}					% Citeringer til bøger mm. \citep
\usepackage[pdftex]{hyperref}						% Til flotte links
\usepackage{graphicx}								% Diverse billed sjov
\usepackage{caption}
\usepackage[labelformat=simple]{subcaption}
\renewcommand\thesubfigure{(\alph{subfigure})}
\usepackage{multirow}								% Multirow in table

\usepackage{mathrsfs}
\usepackage{amssymb}								% Math font
\usepackage[danish=quotes]{csquotes}        		% Benytte for at kunne lave " tegnet
\usepackage[table]{xcolor}							% Farvekoder til tabeller
\usepackage{pdfpages}                               % pakke til at indsætte pdf
\usepackage{titlesec}
\usepackage{rotating}
\usepackage{algorithmic}
\usepackage{algorithm}
\usepackage{tabularx}
\usepackage{enumitem}
\usepackage{alltt}
\usepackage{appendix}
\usepackage[table]{xcolor}
\usepackage{setspace}
	\titleformat{\section}{\normalsize\bfseries}{\thesection}{1em}{}
	\captionsetup{margin=25pt,font={footnotesize,sf},format=hang}
\usepackage{tikz}
\usetikzlibrary{positioning,matrix,arrows}

\usepackage{todonotes}
%New colors defined below
\definecolor{codegreen}{rgb}{0,0.6,0}
\definecolor{codegray}{rgb}{0.5,0.5,0.5}
\definecolor{codepurple}{rgb}{0.58,0,0.82}
\definecolor{backcolour}{rgb}{0.95,0.95,0.92}

%Code listing style named "mystyle"
\lstdefinestyle{mystyle}{
  %backgroundcolor=\color{backcolour},
  commentstyle=\color{codegreen},
  keywordstyle=\color{magenta},
  numberstyle=\tiny\color{codegray},
  stringstyle=\color{codepurple},
  basicstyle=\footnotesize,
  breakatwhitespace=false,
  breaklines=true,
  postbreak=\raisebox{0ex}[0ex][0ex]{\ensuremath{\color{red}\hookrightarrow\space}},
  captionpos=b,
  keepspaces=true,
  numbers=left,
  numbersep=5pt,
  showspaces=false,
  showstringspaces=false,
  showtabs=false,
  tabsize=2,
  escapeinside={(*@}{@*)}
}

%"mystyle" code listing set
\lstset{style=mystyle}

\usepackage[
    backend=biber,
    style=numeric-comp,
    sorting=ynt,
    date=iso8601,
    urldate=iso8601
]{biblatex}
\addbibresource{bib.bib}

%%%%%%%%%%%%%%%%%
%%% PAGESETUP %%%
%%%%%%%%%%%%%%%%%

%\chapterstyle{article}											% Article layout
%\def\thesection{\thechapter.\arabic{section}}           		% 1.1. Ønskes 1.A, ændr arabic til Alph
%\settrimmedsize{297mm}{210mm}{*}								% Tilpasser Layout til a4
%\setlength{\trimtop}{0pt}
%\setlength{\trimedge}{\stockwidth}
%\addtolength{\trimedge}{-\paperwidth}
%\settypeblocksize{634pt}{448.13pt}{*}
%\setulmargins{*}{*}{*}											%Sætter margen i siden til "ryk for siden til side funktion"
%\setlrmargins{*}{*}{*}											%Tilsvarende ovenstående
%\setmarginnotes{17pt}{51pt}{\onelineskip}
%\setheadfoot{\onelineskip}{2\onelineskip}
%\setheaderspaces{*}{2\onelineskip}{*}
%\checkandfixthelayout
%\setlength{\parindent}{0 pt}
%\makepagestyle{rapport}											% Laver en ny type sidehoved
%\aliaspagestyle{chapter}{rapport}								% Sætter 'chapter'-sidehoved = 'ah'-sidehoved
%\addtolength{\voffset}{-0.5cm}
%\setlength{\headsep}{24 pt}
\hyphenation{spej-let mel-lem temperatur-udviklinger hash-tabel hash-tabellen søgefore-spørgsler flyt-te}
% Hvis ord bliver delt forkert, kan de indskrives med den rigtige orddeling her.


\usepackage{tikz}
\usetikzlibrary{shapes}
\usepackage{tikz-qtree,tikz-qtree-compat}
\usepackage{tipa}


%%%%%%%%%%%%%%%%%%
%% STYLE SETUP %%%
%%%%%%%%%%%%%%%%%%

%\pagestyle{rapport}					    						% Sætter 'rapport' til standard-sidehoved
\renewcommand{\chaptermark}[1]{             					% Opretter kommando til kapitelnavne i header
    \markboth{\chaptername\ \thechapter.\ #1}{}}            	%Header og footer setup

%\makeevenhead{rapport}{\leftmark}{}{\theauthor}
%\makeoddhead{rapport}{\theauthor}{}{\titel}
%\makeevenhead{rapport}{\theauthor}{}{\titel}
%\makeheadrule{rapport}{\textwidth}{1pt}
%\makeoddfoot{rapport}{}{}{\thepage}
%\makeevenfoot{rapport}{\thepage}{}{}

\defbibheading{}{\bibsection} % \bibsection is from memoir


%%%%%%%%%%%%%%%%%%%%%%%%%%
%%%   PDF/EPS script   %%%
%%%%%%%%%%%%%%%%%%%%%%%%%%

\usepackage{epstopdf}					    					% Pakke der konverterer EPS til PDF
\graphicspath{ {figures/} }
\DeclareGraphicsExtensions{.png,.pdf,.jpg,.bmp}
