\chapter{Introduction}
\label{chap:Introduction}
This chapter introduces IT-minds, which the project is made in collaboration with.
It also explains why the project is made, and defines the problem definition.
Lastly it explains the structure of the report.
\todo[inline]{For kort problem definition}
\todo[inline]{særskilt udviklings metoder}
\todo[inline]{før problem definitionen vision (5-6 linier)}
\todo[inline]{udviklings metode 0,5 - 1}
\todo[inline]{inkluder billeder af system}
\todo[inline]{list mål som kapitler refererer til, 2-3 per kapitel}

\section{IT-minds}
\label{sec:IT-minds}
IT-minds is a Danish software consultant company\cite{IT-minds}, who primarily employ students
to solve the consultant tasks they get. The idea is that students of a software oriented
study is capable of making software for companies, but may lack the experience to get hired
after the education is done. On top of that since the developers are under education
the customers don't have to pay as much as for someone who is done with their study,
and therefore some money can be saved.

\todo{Is this relevant at all?}
It started in Århus, and later an office was opened in Copenhagen, this means that systems
used internally cannot be based on a local network without a setup like a VPN, and as
the company does not host anything themselves there are no servers to host a system with
a VPN. This means that most systems are based on internet solutions, with separate user login.

Many of the customers of IT-minds wants things made in C\# as that allows then to
take over the code afterwards, and it means that they can use their existing echo
systems. This results in a lot of the developers in IT-minds knowing C\#, or
quickly pick it op. For this reason I have chosen to develop in the same language,
allowing other developers to take over and maintain it later.

\section{Motivation}
\label{sec:Motivation}
Companies that rely on selling to other companies generally need a way to keep track of
what customers they have, and which customers they can potentially have in the future.
IT-minds is an example of such a company. They do have a system that keeps track
of customer relations already, but it does not support all the use cases that they want,
and cannot give the information they are interested in.
Because of this they have asked me to make a POC\footnote{Prove of Concept} for a new
system to handle these cases.

A CRM\footnote{Customer Relation Management} system aims to allow a user to
keep track of what business agreements are currently active, as well as what
could in future be if all goes as planned. It can also allow the user to keep track of who
to contact under a customer, in case the customer is a company, or an organization.

Such a system could be run locally, but that reduces its usability, in that it will
be only a single person in the company, or a single machine that will be able to keep
track of the given information. Because of this a CRM is often run as a webserver, either
on the local network of the company, or on the internet so that it can be accessed
from anywhere, which can be useful if the company has multiple departments, or if the
user wants to access it from a phone, or some other device that may not be connected
directly to the network the CRM is hosted on, or have access to a VPN.

It is not uncommon for a sales department to want to look up the contact information
of a client when they are not in the office, and for a company such as IT-minds
with more than one department the need for a distributed system is needed, this would
generally be on a webserver accessed via the internet.

Even after we have figured out that we want the system to be run on a webserver we still have
different ways to go, each with their own advantages and disadvantages. Some of them
could be writing a static website that is closely connected to the server, so the server
generates the appropriate html allowing for communication with the database, via technologies
such as php, JSP, or ASP.
Another way could be to write a web-api, serving the information only in raw format
on designated paths. A third option could be to make a SOAP\footnote{Simple Object Access Protocol}
web service, and access the information via that.
All this topic is discussed in further detail in chapter~\ref{chap:Technology}.

\section{Development methods}
\label{sec:Development methods}


\section{Vision}
\label{sec:Vision}


\section{Problem definition}
\label{sec:Problem definition}
The goal of this thesis is to create a POC of a CRM, that solves some of the issues
that IT-minds has with their existing CRM, these problems are, amongst others:

\begin{itemize}
  \item Distribution of deal resource usage
  \item Moving a contact from one company to another without loosing history
\end{itemize}

On top of this the system must support the standard things required for a CRM to be useful,
having multiple users, and customers. Because a customer can be a company, and
there can be multiple contact people in one company it is also important that this
is supported. Along with having deals assigned to a customer, so that one can keep track
what deals has been made, and possible future deals to be made.

To do this a full analysis of what is needed in order to create the ideal system
will be created, and relevant technologies will be analyzed. Because IT-minds primarily
works with C\# this will be the language the CRM will be developed in, but C\# supports
many different technologies, and an analysis of these will be used to find the best suited.

After the technologies have been decided the design of the data model will be made, and the design of the system structure will be made. While this is done the implementation will run as well, to run in an agile fashion according to SCRUM.

During the development unit tests will be made, and run against the code enforcing higher quality.
And end-to-end tests will be run as well in order for the flow to be tested, as well as caching cases that was missed while writing the unit tests, and these cases will be implemented as unit tests afterwards.

\section{Goals}
\label{sec:Goals}


\section{Thesis structure}
\label{sec:Thesis structure}
Here I will go over the general structure of the report, and what each of the chapters will focus on.\\

\textbf{Chapter~\ref{chap:Domain analasys} - \nameref{chap:Domain analasys}} will describe the domain the solution is under, including requirements from IT-minds to the solution, and a prioritized order of features that may be implemented into the system. The chapter will also set up some use cases that describe what a user might do in the system, and what should be supported by the final product.\\

\textbf{Chapter~\ref{chap:Technology} - \nameref{chap:Technology}} will look at some of the technologies that are available for developing the system, and make a decision from the findings of this on what technologies will be used to implement the system it self.\\

\textbf{Chapter~\ref{chap:Design} - \nameref{chap:Design}} will focus on the design of the system itself, including the database design, and the architecture of the system. The goal is to design the a structure on both parts so that further development will be easy later on.\\

\textbf{Chapter~\ref{chap:Implementation} - \nameref{chap:Implementation}} will discuss the implementation itself, including the design patterns used and the technologies that was chosen based on the previous chapter.\\

\textbf{Chapter~\ref{chap:Test} - \nameref{chap:Test}} will describe the testing of the system, both unit tests, and integration tests based on the use cases.\\

\textbf{Chapter~\ref{chap:Conclusion} - \nameref{chap:Conclusion}} will hold the final product up against the requirements, and a conclusion will be drawn from the project. Future development of the product will also be discussed in this chapter.
