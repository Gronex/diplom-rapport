\chapter{Introduction}
\label{chap:Introduction}

\section{IT-minds}
\label{sec:IT-minds}
IT-minds is a \todo{capitalized or no?}Danish software consultant company, who primarily employ students
to solve the consultant tasks they get. The idea is that students of a software oriented
study is capable of making software for companies, but may lack the experience to get hired
after the education is done. On top of that since the developers are under education
the customers don't have to pay as much as for someone who is done with their study,
and therefore some money can be saved.

\todo{Is this relevant at all?}
It started in Århus, and later an office was opened in Copenhagen, this means that systems
used internally cannot be based on a local network without a setup like a VPN, and as
the company does not host anything themselves there are no servers to host a system with
a VPN. This means that most systems are based on internet solutions, with separate user login.

Many of the customers of IT-minds wants things made in C\# as that allows then to
take over the code afterwards, and it means that they can use their existing echo
systems. This results in a lot of the developers in IT-minds knowing C\#, or
quickly pick it op. For this reason I have chosen to develop in the same language,
allowing other developers to take over and maintain it later.

\section{Motivation}
\label{sec:Motivation}
\todo{mention it is prove of consept}
Companies that rely on selling to other companies generally need a way to keep track of
what customers they have, and which customers they can potentially have in the future.

IT-minds is one such company. They do have a system that keeps track of customer relations,
but it does not support all the use cases that they want, and cannot give the information
they are interested in.

A Customer Relation Management system, or CRM for short aims to allow a user to
keep track of what business agreements are currently active, as well as what
could be if all goes as planned. It can also allow the user to keep track of who
to contact under a customer, in case the customer is a company, or an organization.

Such a system could be run locally, but that reduces its usability, in that it will
be only a single person in the company, or a single machine that will be able to keep
track of the given information. Because of this a CRM is often run as a webserver, either
on the local network of the company, or on the internet, so that it can be accessed
from anywhere, which can be useful if the company has multiple departments, and if the
user wants to access it from a phone, or some other device that may not be connected
directly to the network.

It is not uncommon for a sales department to want to look up the contact information
of a client when they are not in the office, and for a company such as IT-minds
with more than one department the need for a distributed system is needed, this would
generally be on a webserver.

Even after we have figured out that we want the system to be run on a webserver we still have
different ways to go, each with their own advantages and disadvantages. Some of them
could be writing a static website that is closely connected to the server, so the server
generates the appropriate html allowing for communication with the database, via technologies
such as php or JSP or ASP.
Another way could be to write a web-api, serving the information on designated paths
only serving the raw data. This topic is discussed in further detail
in chapter~\ref{chap:Technology}.

\todo{subtract reasons for new CRM}
This project is made in collaboration with IT-minds which is a software consultant company,
who is in need of a new CRM to keep track of their customers. The most basic need is
the ability to have companies stored with deals, and having a person from sales assigned
to a deal. It is also important to be able to divide a deal into multiple small parts
so the income from the deal can be throttled up or down for a time, during the lifetime
of the deal. This is one of the reasons IT-minds decided that they need a new CRM in the
first place.


\section{Problem definition}
\label{sec:Problem definition}

\section{Thesis structure}
\label{sec:Thesis structure}
