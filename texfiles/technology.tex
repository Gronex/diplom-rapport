\chapter{Technology}
\label{chap:Technology}
In this chapter I will be looking at what technologies to use, and compare the
ones I choose with some other alternatives, in order to show why the ones I have
settled on makes sense for the task I am doing. 

The technologies I will be looking into revolves around how the data will be
served to the user, and how I will be persisting the data the system is going to
handle. In the end of the chapter I will explain the technology I have decided
to go with for the front-end, but because it is only meant as an example of how a
front-end can be made on top of the back-end and as such there will not be too
many comparisons.

\section{Serving}
\label{sec:Serving}
I can choose to serve the data in multiple ways, even after I have decided that
the the user will be visiting a website to get the data. I can go with static
serving which is what Section~\ref{sub:Static_serving} will focus on, or I
could go with a web service, and then go with a single page application or
similar technology to present the data, which is what
Section~\ref{sub:web_service} will discuss. 

\subsection{Static serving}
\label{sub:Static_serving}

One way to implement the website could be via purely static serving. This means
that the server will receive a request from the client to which the server then
finds the appropriate HTML, CSS and potentially JavaScript files to give back,
while first inserting values for representing the data in the database, and the
browser then renders this for the user. 

Because of our requirements in regards to user verification and database
dependent data I cannot go with the purest static serving model. I can however
use technologies such as php, ASP.NET, or spring. Two of which are not C\#,
which was another goal for the project, to support maintainability within IT
minds. 

Sprint is based on the JVM\cite{spring-framework}, which means that you would need to run in languages that uses this virtual machine in order for it to work. Some languages that could use Spring would be Java or Scala, where most in IT minds would be able to develop in Java, since it is similar to C\# on a lot of points. And most of the people working in IT minds have build things in Java before.

If I were to create the service with Java I could support different operating systems to host the server out of the box, which can be convenient when trying to find cheap hosting.

For the most part the universal support for hosting is also true for php.\ php is
a server scripting language\cite{php-home}, designed to support web development
well. It can be integrated into ones HTML pages, and from there insert new HTML.\
The user will never see the logic of php, like they would never see it for
Spring, as when the request is made to the server the code is evaluated, and the
template is pre-rendered, which means the results of the template values are
inserted, and what the client gets is only the pure HTML.\ This can be an
advantage since I then don't show any internal logic to the user. 

ASP.NET work much in the same way that Spring does except it runs on the .NET
virtual machine made by Microsoft, A few different languages run on the .NET
virtual machine, similar to different languages running on the JVM.\ One of these
languages is C\# which is very popular language in IT minds, and among their
customers. 

ASP can also use the template option, where they render some code on the server
before it is sent to the user, who will have no idea which parts are hard coded
and which are dynamic. 

In previous versions of .NET the user was mostly locked on to the windows
platform, with a few exceptions like mono\footnote{Read more about the mono
  project at: http://www.mono-project.com/}. But recently Microsoft has decided
to start supporting other platforms, as part of their open sourcing of parts of
the .NET platform\cite{.net-core}, in a push to make ASP.NET Core 1 more modular
and open. 

\subsection{Web service}
\label{sub:web_service}
Another option could be to create a web service, that will expose the data
in a way for clients to call to, allowing for multiple different front-ends to be
implemented, with the web service as the way of getting hold of the data. 

The advantage of using a service for a setup such as this one, would be that if
for some reason the website needs to be updated with a new look, it can be
done without ever touching the back-end. Another advantage is that I can make
multiple different clients communicating with the same service, meaning I can
decide later on that I also want to support a mobile application. This can be
done entirely from an exposed interface, without the mobile developer ever
having to look at the backing code, only the documentation of the interface. 

For webservices there are a couple of different types, two of which I will discuss
in the following sections. 

\subsubsection{SOAP}
\label{subs:SOAP}
SOAP\footnote{Simple Object Access Protocol} is a type of web service which
relies on sending data in XML.\ 

The goals of SOAP is to be extensible and simple at the same
time\cite{soap:messaging}. This means that a lot of features that is often used
in order for the messaging to work correctly such as correlation and routing is
not part of the basic SOAP messaging protocol, but the protocol is made so that
it can support them with extensions. An advantage of this is also that SOAP is
not bound to a specific protocol for sending the data, and can therefore be
implemented to use whatever is best suited for the situation. 

In soap you send all the data in an XML envelope, containing an optional header
section and a body section. The header section is for data that extends the
basic SOAP protocol, such as information of priority or correlation between
messages\cite{soap:messaging}. This section is there to make the entire protocol
modular, and extensible. Where the body is for the message itself. 

The body of a SOAP message is where the data is contained. In a successful
scenario the body contains whatever was expected for the client to get. But it
can contain a fault as well, meaning something went wrong in the transaction.
The fault holds information on what type of error happened, so that it can be
handled. The faults are identified with fault codes, which can hold sub codes to
specify deeper what the problem in this particular instance was. These fault
codes are part of the application domain, so a code could be ``NoCompany'' and a
client receiving that code would then know exactly what went wrong, which in
this case would be that there is no company available. 

The primary argument against using SOAP is that the initial front-end will be a
website, which means that I will have to either combine one of the static
serving ways with the SOAP service, or have to have JavaScript call the service.
If I go with a setup where I have JavaScript call the service the JavaScript has
to use some extra energy on translating the XML into the data structures of
JavaScript, which is JSON.\

\subsubsection{REST}
\label{subs:REST}
An alternate type of web services is a REST API, which stands for
Representational state transfer. A REST API generally communicate via the HTTP
protocol. This limits a little compared to SOAP which can work with any protocol
as described in Section~\ref{subs:SOAP}. 

The appeal of a web API is the simplicity, as the HTTP protocol is so well
established, with error codes for most situations, and the added ability to give
reasons on top. Another reason is that since it runs on HTTP it makes sense to
use the often already set up rules for firewalls and so forth, to allow calls to
websites to go through. 

The body of the simplest call is limited to only a verb and the URL to which you
want to call, where in SOAP you would need to set up an entire request body. 

As we surf the web we can get all sorts of different data types, all the way
from a simple text file to a binary file. This is something REST API's allow as
well, which means that instead of having the resource locked to come in XML
format you can send it as eg. JSON, or some other format that
fits\cite{rest:basic}, the client can even choose a data type if the API
supports that type. Because you can never really be sure that the caller is not 
malicious and try to format or break the communication by changing the sent data
the type safety that XML gives is not as important as you would have to verify
everything anyways. 

As with SOAP in REST you can have headers carry extra data, such as what format you
would like the response to be in, and the token for authenticating the user.
This is a feature supported and used by standard HTTP web servers and browsers when
we surf the internet, so REST just decided to keep using this as well. 

There are some rules for a web API to be called RESTful\cite{rest:msdn} needs to
fulfill. One of these rules is that the paths have to follow some fairly simple
rules. If I want to get a list of eg.\ activities I would call a path that ends
with ``/activities'' with the verb ``GET''. If the RESP API has a resource type on
the path activities it will return a list of all of them with the status code 200
which stands for OK.\ When I then want a specific one I call ``/activities/:id''
with the verb ``GET'' where :id is the id of the activity I want. 

This is only the read part of CRUD, but the other parts are also supported. They
just use different verbs, the most common are ``POST'' for create, ``PUT'' for
updating, although ``PATCH'' could also be used for that, and ``DELETE'' for
deleting an element. If it is an existing resource you are operating on, the
address should end in the id of the resource to be worked on, and otherwise just
in the resource type, such as the first get. 

Another rule for a service to be called REST is that it has to be state less,
which means that one request cannot be dependent on another to be done first, as
the system is not allowed to keep the state of a call. If I want to update a
resource I am allowed to get the resource first, but it should not be required,
and we cannot know if someone else have removed the resource between us getting
it and trying to update it. Because of this it is important to return the right
error codes when something goes wrong, as it is then clear what went wrong. 

The important thing is that within one connection I do not make multiple calls
to the server, as that would mean that I have a connection state. The client is
allowed to have a state of its own, in fact the client does not have to abide by
any of the REST rules for it to communicate with a REST service, except for in
the communication itself. The server will often keep a state as well in that the
data is often stored somewhere which means that the data is in a state. 

Another rule is that the resources must be cacheable, meaning that the protocol
holding the data must have some way of allowing caching. Since HTTP is the most
common protocol this is generally done with the cache-control
headers\cite{rest:elkstein:architeture}. 

Lastly the system must be layered\cite{rest:uci}, meaning that it is only
allowed to know its immediate neighbors. Say a REST service needs to pull data
from another service that is pulling from some legacy service, the REST service
is not allowed to skip the first service because this would mean that it is
communicating with a layer deeper than its immediate neighbor. The reason for
this is that it cannot end up in a situation where we have the middle layer eg.
employing some extra security before the last layer that is now skipped. It also
enforces some organizational bounds, and keeps the service structure clean. 

\subsection{Chosen type }
\label{sub:Chosen type}
Based on the arguments for and against the different technologies, and the
requirement that IT minds needs to be able to continue development after this
project is done. And a little bit of my own interest, I have come to the
conclusion that the best way of serving the data is via a RESTful API.\ There are
a couple of reasons for this conclusion. First of which is the possibility for
me to use the language I see best fit. Another reason is also that a CRM could
be useful as a mobile app at a later time, and that would need some API to talk
to in order for it to work with the same back-end as the website. 

The reason of going with a RESTful architecture instead of SOAP is two parted.
First off in IT minds most of our customers are looking for RESTful more often
than SOAP, since it is easier to work with, leading to the developers having
more experience in developing such a system, myself included. 

The other reason is that JSON is easier to parse than XML in a lot of languages,
and is the data structure of  JavaScript which is going to be what will be
driving the logic of the web front-end, so the front-end will not need to parse it
at all. 

A table of some of the pros and cons of the technologies I have discussed until
now have been set up in table~\ref{tab:serving-pro-con}. 

\begin{table}[h]
  \resizebox{\linewidth}{!}{%
    \begin{tabular}{l|l l}
      & Pros & Cons\\
      \midrule
      \textbf{Static serving} & Dynamic content & Single front-end \\
      & Hidden code & Mixing of front and back-end code\\
      \midrule
      \textbf{SOAP} & Possible multiple front-ends & XML\\
      & Extendable & Not many consuming libraries for web\\
      & Custom error types & \\
      & Dynamic content\\
      \midrule
      \textbf{REST} & Possible multiple front-ends & Easy to break standards\\
      & Not locked to singe data format   & \\
      & Well known protocol               & \\
      & Support for caching               & \\
      & Dynamic content\\
    \end{tabular}
  }
  \caption{Pros and cons of different ways of serving the data}
\label{tab:serving-pro-con}
\end{table}

The implementation of the RESTful service will be made using ASP.NET since I
have experience in it and it is well suited for such a task. It will however
limit the systems it can run on, as I will not be using ASP.NET Core 1 which can
run on systems with Linux on them\cite{asp5:intro}. 

The reason I have chosen not to go with Core 1, but Webapi 2, which still runs
on ASP.NET is because Core 1 is still only in release
candidate\cite{asp_core_1:roadmap}, and to my experience it is still missing
some of the features related to authentication on a web API setup, which will be
required in this project. 

\section{Database }
\label{sec:Database_engine}
When it comes to which database to use, there are a lot to take into account.
First off I need to look at what type of database I want. When I have decided
this, I need to figure out which engine for the database I am going to go with.

This analysis will not be exhaustive as there are a lot of different types, but
I will look at some of the most common ones and compare them. 

\subsection{Relational database}
\label{sub:Relational database}
One type of database is the relational database. This type of database has all
the data stored in tables, where there will be some sort of key, allowing a
table to be connected to another. The advantage of this is that they can reduce
a lot of redundant data by storing in different tables, and then having multiple
entries pointing at the same column\cite{ibm:choosing_dbms}. 

An important aspect when working with relational databases is the idea of
normalization. This is a way to reduce redundancy, and thereby the likeliness
of inconsistent data. 

\subsection{Object database}
\label{sub:Object database}
This form of database is designed to resemble the way data is structured in
object oriented programming languages such as Java, C++ or C\#. This means that
the data is structured in classes, that can inherit from one another. The idea
is that the API to the database makes sense for object oriented languages. 

The risk with using this form of database is that it is easy to fall into the
same design patterns as one would in the application itself, which may not be as
efficient when it comes to a database, as they may require more resources, or
time\cite{ibm:choosing_dbms}. 

\subsection{Document database}
\label{sub:Document database}
A document database is a database that instead of saving in tables it saves the
data in documents. The structure can vary, but initially most of these databases
stored their data in XML\cite{ibm:choosing_dbms}. Now JSON and variants there
off has started to show up for some database systems. 

The advantage of this form of database is the lack of structure, as it allows
for a more accepting storage for the users. The primary drawback of document
databases is that they are generally not as efficient as other database systems,
they take up more space, and the queries can be slow to set
up\cite{ibm:choosing_dbms}. 

It can be a good type of database for some web applications, especially if some
parts of the database is kept structured, but it requires extra validation logic
from the application when dealing with the data. 


\subsection{Decision }
\label{sub:database_decision}
In table~\ref{tab:sedatabse-pro-con} some of the advantages and disadvantages
are compared against one another, to make it easier to get an overview of the
pros and cons. 

For this project I have decided to use a Relational database, as it has fast
queries, and the structure enforces a valid model. Since the application is
developed as a web API in C\# using ASP.NET it makes sense to use MsSQL for the
specific engine, as this is a native supported database in the entity
framework, which is the recommended data-access framework when using C\#
according to Microsoft\cite{entity:microsoft}. The entity framework is an
ORM\footnote{Object Relational Mapping} framework, which means it maps the
results from the database to C\# objects. One thing to be aware of when
using the entity framework is that if using code first I still risk falling into
the same design patterns of the Object database, to avoid this I will design the
database like I would if I were to design a database and then create it
using SQL.\ The creation of the database will still be done using code first but
the classes will map from my ER diagram to the code classes. This way I should
avoid some of the worst pit falls of using code first.

MsSQL also makes sense as it is proven to work for others in a lot of scenarios,
based on its usage statistics\cite{obdms:gartner}. 

\begin{table}[h]
  \resizebox{\linewidth}{!}{%
    \begin{tabular}{l|l l}
      & Pros & Cons\\
      \midrule
      \textbf{Relational database}
      & Data normalization & Harder to update data structure when live\\
      & Low to no redundant data & \\
      \midrule
      \textbf{Object database}
      & Easy relation to OOP\protect\footnotemark{} & Can lead to bad data designs\\
      \midrule
      \textbf{Document database}
      & Good for often changing data structure & Use more space\\
      & & Often slower than other database types
    \end{tabular}
  }
  \caption{Pros and cons of different types of databases}
\label{tab:sedatabse-pro-con}
\end{table}
\footnotetext{Object Oriented Programming}

\section{Frontend}
\label{sec:Frontend}
For the front-end there are a few different ways I can go. Because of my
decision to go with a web API for the back-end I need some dynamic way of
handling the remote calls to the server. 

Because I am working in a browser it makes sense to go with JavaScript, as this
is a scripting language supported by most modern browsers. Even in this I still
have a lot of options as to how to handle the calls. There are a lot of options
for frameworks as well. Which allow me to develop with different structures,
such as MVVM\footnote{Model View ViewModel} and MVC\footnote{Model View
  Control}, and some others. 

Some of the options that follow the MVC model is AngularJS and EmberJS.\ Angular
is made by Google\cite{angularjs}, and allow me to separate the application out
into separate parts, having a controller per view. I can also separate functions
into services, and create directives and components, which are view elements,
that allows me to reuse elements for the website in different places on the site
without having to rewrite the code over and over again. 

Ember allows for many of the same things as angular, but has not been around as
long, and handles the things slightly differently\cite{emberjs}. 

They both have a templating language implemented to allow integration, and
data binding with the HTML view. This allows both for the data to live in and be
updated by the JavaScript, as well as having control structures in the HTML such
as loops and conditionals, to allow for a more dynamic experience without having
to deal directly with JQuery or the document object of the window object. 

Because my focus in this project is not on the front-end, but to make the back-end
so that different front-ends will be able to use it I have decided to go with
AngularJS as that is the one I have the most experience with, and it has a lot
of modules build by different people that can be used to make the experience a
little better, it is also a mature framework build by a company that specializes
in the web. 

\section{Chapter summary }
There are a couple different ways to go in regard to technology with a project
like this one. One of the first limiting decisions could be the language, where
I have chosen to go with C\# because it is one of the most used languages in IT
minds, and since the project is for IT minds it will be them who will continue
work on it at a later date.

I have decided to go with a webservice running in C\#. More specifikally
ASP.NET with Webapi 2 serving JSON.\ This allows me to not worry about the
front-end when developing the back-end, as a JSON API running over the HTTP
protocol can be used both by a web application, as well as an application for a
phone, or a native desktop application, without any of them knowing the other exists.

For data storage I have gone with MsSQL as it is a relational database, which is
the type I have concluded fits best for my project. It also have good support in
the .NET environment, as both are developed by Microsoft. For accessing the
database I will be using an ORM by the name of Entity framework, also developed
by Microsoft. This will allow me to not worry about SQL injection, and treat the
database entities more or less like C\# objects.

I will also be developing a front-end as an example of how that part of the
system could look, as well as to make sure that my decisions on the back-end
actually makes sense in a user interactive way. For this i have decided
to go with AngularJS in a JavaScript single-page-application, mostly because it is
a mature framework that I have experience using.
