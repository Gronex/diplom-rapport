\chapter{Domain analasys}
\label{chap:Domain analasys}

\section{Requirements}
\label{sec:Requirements}
In order for me to know what is expected of the system I have spoken with Kristian, and together we have come up with some requirements for the system. This includes what users should be in the system, as well as what those users can do. We also talked about some of the views that are important to have for the CRM to be usable, and what annoys him in the existing CRM's, so that this one will be more to IT-minds needs.

\subsection{Users}
\label{sub:Users}

One of the basic requirements are the users that the system will support. Ideally the following three user types should be supported, each with the rights of the previous, as well as some rights of their own.

\begin{enumerate}
  \item Super
  \begin{itemize}
    \item Invites new users\todo{should not be here, but in text}
    \item Defines what role users are in
    \item Defines what groups the user is in
  \end{itemize}
  \item Administrator
  \item Standard user
\end{enumerate}

\todo[inline]{Explain the base idea with each usertype}

\subsection{Resources}
\label{sub:Resources}

A theme with a lot of the requirements was the desire to be able to customize, instead of having the system enforce things on the users. The things that should de customizable is categories, departments, stages and the like. Generally things that tag something, and that may carry a value with it, such as stage, which is supposed to have a percentage of certainty of an opportunity. Letting the user make the stages them selves also allows them to set what the stage actually means, allowing a system that suites the individual company's way of working.

The customization should only be something that the super administrator can create, as to avoid too many things that essentially mean the same to show up. In the case of an opportunity the tag that a stage shows may be fitting while the value is not, it is also desired that the creator of the opportunity can overrule the suggested value that the stage carries to fit better with each individual opportunity.



An opportunity can be edited at any time, by the creating user.
Opportunities:
\begin{itemize}
  \item Associated with user
  \item Optionally associated with contact
  \item Associated with company
  \item Start and end date
  \item Price
  \item Stage
  \begin{itemize}
    \item Custom made by super user
    \item associated with percentile
  \end{itemize}
  \item Percentile (overriding stage percentile)
  \item Category
  \begin{itemize}
    \item Made by super user
  \end{itemize}
  \item Price per hour
  \item Development department\todo{Different name to support broader amount of people?}
  \item Group (taken from the creating user, at the time of creation)
  \begin{itemize}
    \item Assigned and created by super user
  \end{itemize}
  \item Expected close\todo{??}
  \item Addition of files
\end{itemize}

Activities:
\begin{itemize}
  \item Custom types with custom weighting value
  \item Associated with user
\end{itemize}

\subsection{Visualization}
\label{sub:Visualization}


\section{Chapter summary}
\label{sec:Chapter summary}



\todo{update after meeting with kristian}

\begin{itemize}
  \item Users
  \begin{itemize}
    \item Sales people
    \item Administrators\todo{Figure out what they can that sales personnel cannot}
  \end{itemize}
  \item Creation of companies/customers
  \begin{itemize}
    \item (low priority) Nesting of companies
  \end{itemize}
  \item Creation of deals under company
  \item Creation of person under company
  \begin{itemize}
    \item Move person to new company, but letting old relations stay in old company
  \end{itemize}
  \item Assignment of person/people from company to deal
  \item Assignment of salesperson to company
  \item Addition of resource income from deal
  \begin{itemize}
    \item Separation of income into periods
    \item Overview of income per
    \begin{itemize}
      \item Per period
      \item Per salesperson
      \item Per company
    \end{itemize}
  \end{itemize}
  \item Creation of tasks
  \begin{itemize}
    \item Can be associated with company
    \item Can be associated with person in company
    \item (low priority) Integration with Google calendar
  \end{itemize}
\end{itemize}
