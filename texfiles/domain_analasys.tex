\chapter{Domain analasys}
\label{chap:Domain analasys}
\todo[inline]{introduce chapter}

\section{Requirements}
\label{sec:Requirements}
In order for me to know what is expected of the system I have spoken with Kristian, and together we have come up with some requirements for the system. This includes what users should be in the system, as well as what those users can do. We also talked about some of the views that are important to have for the CRM to be usable, and what annoys him in the existing CRM's, so that this one will be more to IT-minds needs.

\subsection{Users}
\label{sub:Users}

One of the basic requirements are the users that the system will support. Ideally the following three user types should be supported, each with the rights of the previous, as well as some rights of their own.

\begin{enumerate}
  \item Super
  \begin{itemize}
    \item Invites new users\todo{should not be here, but in text}
    \item Defines:
    \begin{itemize}
      \item User groups
      \item Categories
      \item Departments
      \item States
      \item Activity categories
    \end{itemize}
  \end{itemize}
  \item Administrator
  \item Standard user
\end{enumerate}

\todo[inline]{Explain the base idea with each usertype}

\subsection{Resources}
\label{sub:Resources}

A theme with a lot of the requirements was the desire to be able to customize, instead of having the system enforce things on the users. The things that should de customizable is categories, departments, stages and the like. Generally things that tag something, and that may carry a value with it, such as stage, which is supposed to have a percentage of certainty for an opportunity. Letting the super user make the stages them selves also allows them to set what the stage actually means, allowing a system that suites the individual company's way of working better.

The customization should only be something that the super administrator can create, as to avoid too many things that essentially mean the same to show up. In the case of an opportunity the tag that a stage shows may be fitting while the value is not, therefore it is also desired that the creator of the opportunity can overrule the suggested value that the stage carries to fit better with each individual opportunity.

The system should keep track of opportunities which are possible deals, and deals that are won. An opportunity needs to contain some information such as who is responsible for it, and who the deal is with, as well as how much it will earn. A full list of what an opportunity should contain can be seen in appendix~\ref{app:Oppotunity}.

The primary reason for IT-minds to want a new CRM is that most other CRM's frontload the earning from an opportunity, resulting in a task that that will earn a lot of money, but also takes up a lot of resources looking like it has no impact on any of the months after the first. This is not how it works in reality, as the company who hires ex. IT-minds to work for a couple of months pay a little bit every month instead of everything all at once. The solution to this is mostly in regards to data visualization, where we need to keep track of the span of an opportunity, and what it will earn over that period, this should then be divided out over the span of the opportunity, also meaning that the earnings of a user is based on the same, since it needs to be possible to set goals for the user on a per month basis.

Since an opportunity is a bigger thing, and there will be a lot of work related to a single one, we also need a way to keep track of smaller tasks, such as making a call, or going to a meeting. This is where an activity comes in to the picture. An activity should be able to be several things defined by the super admin, but the primary thing that it does is define a task for a user to do. The task could be to send an email, or make a call, or go to a conference. Because these things don't take the same amount of time a category for them can have a value, so it can be visualized what the users are using their time on. This is useful as right now IT-minds ports the number of activity for all the users to a spread sheet, and graphs from there, but if the system could do that for them some inconvenience would be saved.

\subsection{Visualization}
\label{sub:Visualization}


\section{Chapter summary}
\label{sec:Chapter summary}
